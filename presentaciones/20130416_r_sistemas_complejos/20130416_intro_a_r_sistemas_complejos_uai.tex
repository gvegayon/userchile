\documentclass[10pt, letterpaper]{beamer}\usepackage{graphicx, color}
%% maxwidth is the original width if it is less than linewidth
%% otherwise use linewidth (to make sure the graphics do not exceed the margin)
\makeatletter
\def\maxwidth{ %
  \ifdim\Gin@nat@width>\linewidth
    \linewidth
  \else
    \Gin@nat@width
  \fi
}
\makeatother

\IfFileExists{upquote.sty}{\usepackage{upquote}}{}
\definecolor{fgcolor}{rgb}{0.2, 0.2, 0.2}
\newcommand{\hlnumber}[1]{\textcolor[rgb]{0,0,0}{#1}}%
\newcommand{\hlfunctioncall}[1]{\textcolor[rgb]{0.501960784313725,0,0.329411764705882}{\textbf{#1}}}%
\newcommand{\hlstring}[1]{\textcolor[rgb]{0.6,0.6,1}{#1}}%
\newcommand{\hlkeyword}[1]{\textcolor[rgb]{0,0,0}{\textbf{#1}}}%
\newcommand{\hlargument}[1]{\textcolor[rgb]{0.690196078431373,0.250980392156863,0.0196078431372549}{#1}}%
\newcommand{\hlcomment}[1]{\textcolor[rgb]{0.180392156862745,0.6,0.341176470588235}{#1}}%
\newcommand{\hlroxygencomment}[1]{\textcolor[rgb]{0.43921568627451,0.47843137254902,0.701960784313725}{#1}}%
\newcommand{\hlformalargs}[1]{\textcolor[rgb]{0.690196078431373,0.250980392156863,0.0196078431372549}{#1}}%
\newcommand{\hleqformalargs}[1]{\textcolor[rgb]{0.690196078431373,0.250980392156863,0.0196078431372549}{#1}}%
\newcommand{\hlassignement}[1]{\textcolor[rgb]{0,0,0}{\textbf{#1}}}%
\newcommand{\hlpackage}[1]{\textcolor[rgb]{0.588235294117647,0.709803921568627,0.145098039215686}{#1}}%
\newcommand{\hlslot}[1]{\textit{#1}}%
\newcommand{\hlsymbol}[1]{\textcolor[rgb]{0,0,0}{#1}}%
\newcommand{\hlprompt}[1]{\textcolor[rgb]{0.2,0.2,0.2}{#1}}%

\usepackage{framed}
\makeatletter
\newenvironment{kframe}{%
 \def\at@end@of@kframe{}%
 \ifinner\ifhmode%
  \def\at@end@of@kframe{\end{minipage}}%
  \begin{minipage}{\columnwidth}%
 \fi\fi%
 \def\FrameCommand##1{\hskip\@totalleftmargin \hskip-\fboxsep
 \colorbox{shadecolor}{##1}\hskip-\fboxsep
     % There is no \\@totalrightmargin, so:
     \hskip-\linewidth \hskip-\@totalleftmargin \hskip\columnwidth}%
 \MakeFramed {\advance\hsize-\width
   \@totalleftmargin\z@ \linewidth\hsize
   \@setminipage}}%
 {\par\unskip\endMakeFramed%
 \at@end@of@kframe}
\makeatother

\definecolor{shadecolor}{rgb}{.97, .97, .97}
\definecolor{messagecolor}{rgb}{0, 0, 0}
\definecolor{warningcolor}{rgb}{1, 0, 1}
\definecolor{errorcolor}{rgb}{1, 0, 0}
\newenvironment{knitrout}{}{} % an empty environment to be redefined in TeX

\usepackage{alltt}

\usetheme{Warsaw}
\usepackage[buttonsize=1em]{animate}
%\hypersetup{urlcolor=blue}

\title[{\tt R}-UAI]{{\tt R} Doctorado de Sistemas Complejos}
\author[GGV]{George G. Vega}
\institute[SPensiones]{Superintendencia de Pensiones}
\date{\today}

\begin{document}

\frame{\maketitle}

\frame{\frametitle{Agenda}\tableofcontents}




%\SweaveOpts{concordance=TRUE}

%%%%%%%%%% DONDE BUSCAR INFORMACION %%%%%%%%%%
\section{D\'onde buscar informaci\'on}
\frame{\frametitle{Agenda}\tableofcontents[currentsection]}

\begin{frame}
\frametitle{D\'onde buscar informaci\'on}
\framesubtitle{Tutoriales Online}
\begin{itemize}
\item R wiki (oficial) \url{http://rwiki.sciviews.org/doku.php}
\item R-programming (wiki-book) \url{http://en.wikibooks.org/wiki/R_Programming}
\item Quick-R \url{http://www.statmethods.net/}
\item Flowing Data \url{http://flowingdata.com/category/tutorials/}
\end{itemize}
\end{frame}

\begin{frame}
\frametitle{D\'onde buscar informaci\'on}
\framesubtitle{Libros}
\begin{itemize}
\item Norman Matloff, The Art of R Programming (2009) \url{http://heather.cs.ucdavis.edu/~matloff/132/NSPpart.pdf}
\item Introducci\'on a R \url{http://cran.r-project.org/doc/contrib/R-intro-1.1.0-espanol.1.pdf}
\item R para principiantes \url{http://cran.r-project.org/doc/contrib/rdebuts_es.pdf}
\end{itemize}
\end{frame}

\begin{frame}
\frametitle{D\'onde buscar informaci\'on}
\framesubtitle{Otros}
\begin{itemize}
\item R-bloggers: Agregador de blogs sobre R \url{http://www.r-bloggers.com/}
\item Bioconductor: Herramientas para el an\'alisis gen\'etico (biol\'ogico) \url{http://www.bioconductor.org/}
\item CRAN: The Comprehensive R Archive \url{http://cran.r-project.org}
\item R Graph Gallery \url{http://gallery.r-enthusiasts.com/}
\item R Seek: Google de R \url{http://rseek.org/}
\end{itemize}
\end{frame}

%%%%%%%%%% PAQUETES IMPORTANTES EN R %%%%%%%%%%
\section{Algunos Paquetes importantes}
\frame{\frametitle{Agenda}\tableofcontents[currentsection]}

\begin{frame}[fragile]
\frametitle{Paquetes importantes}
\begin{itemize}
\item[ggplot2] An implementation of the Grammar Graphics (Wickham 2013)
\item[lattice] Lattice Graphics
\item[rgl] 3D visualization device system (OpenGL).
\item[knitr] A general-purpose package for dynamic report generation in R.
\item[foreign] Read Data Stored by Minitab, S, SAS, SPSS, Stata, Systat, dBase, ...
\item[igraph] Network analysis and visualization
\item[deSolve] General solvers for initial value problems of ordinary  differential equations (ODE), partial differential  equations (PDE), differential algebraic equations (DAE), and delay differential  equations (DDE).
\end{itemize}
\end{frame}

\section{Must Know de {\tt R}}
\frame{\frametitle{Agenda}\tableofcontents[currentsection]}

\begin{frame}[fragile]
\frametitle{Must Know de {\tt R}}
\begin{itemize}
\item {\tt R} es un lenguaje de c\'odigo interpretado (no compilado como {\tt C C++ Python}, etc). Sigue esquema OOP (Object Oriented Programming), es din\'amico (mayor parte de sus checkeos se realiza en ejecuci\'on).
\item Mayor parte de las funciones de R son transmorficas
\item El comportamiento de las funciones depender\'a de la clase del objeto (puede ser S3 o S4), a estos se les llama m\'etodos
\item para instalar paquetes {\tt install.packages()}
\item para pedir ayuda {\tt ??'prcomp'}
\item Tipos de objetos en R {\tt data.frame, list, matrix, factor}
\item Recordar semillas en Procesos Pseudo-Aleatorios ({\tt set.seed()})
\item {\tt lapply} es preferido a {\tt for()} (velocidad).
\item {\tt R} posee varias rutinas matem\'aticas y estad\'isticas optimizadas (escritas en {\tt C}). En el caso de querer m\'as velocidad, se puede integrar con {\tt C} ({\tt ?.C}), {\tt C++} ({\tt ?.Call}) {\tt Python} ({\tt ?RPy}).
\end{itemize}
\end{frame}

%%%%%%%%%% ALGUNOS EJEMPLOS %%%%%%%%%%
\section{Algunos ejemplos}
\frame{\frametitle{Agenda}\tableofcontents[currentsection]}

%% Animaciones
\begin{frame}[fragile]
\frametitle{Algunos Ejemplos}
\framesubtitle{animaciones}
\begin{knitrout}
\definecolor{shadecolor}{rgb}{0.969, 0.969, 0.969}\color{fgcolor}\begin{kframe}
\begin{alltt}
\hlfunctioncall{library}(animation)
\hlfunctioncall{demo}(\hlstring{"Mandelbrot"}, echo = FALSE, package = \hlstring{"animation"})
\end{alltt}
\end{kframe}



















\animategraphics[width=.45\linewidth,controls,loop]{2}{figure/animate-demo}{1}{20}

\end{knitrout}

\end{frame}


% Graficos 3D
\begin{frame}[fragile]
\frametitle{Algunos Ejemplos}
\framesubtitle{gr\'aficos 3D}
\begin{knitrout}\footnotesize
\definecolor{shadecolor}{rgb}{0.969, 0.969, 0.969}\color{fgcolor}\begin{kframe}
\begin{alltt}
\hlcomment{# Funcion}
fun <- \hlfunctioncall{function}(x, y) \{
    \hlfunctioncall{return}(\hlfunctioncall{sin}(x) - \hlfunctioncall{cos}(x))
\}

\hlcomment{# Graficando}
x <- y <- \hlfunctioncall{seq}(0, 2 * pi, pi/18)
\hlfunctioncall{persp}(x, y, z = \hlfunctioncall{outer}(x, y, fun), col = \hlstring{"blue"})
\end{alltt}
\end{kframe}
\includegraphics[width=.45\linewidth]{figure/persp-demo} 

\end{knitrout}

\end{frame}

%% Analisis numerico (prcomp)
\begin{frame}[fragile]
\frametitle{Algunos Ejemplos}
\framesubtitle{An\'alisis de Componentes Principales}
\begin{knitrout}\footnotesize
\definecolor{shadecolor}{rgb}{0.969, 0.969, 0.969}\color{fgcolor}\begin{kframe}
\begin{alltt}
\hlfunctioncall{data}(USArrests)
x <- \hlfunctioncall{prcomp}(~Murder + Assault + Rape, data = USArrests, scale = TRUE)
\hlfunctioncall{summary}(x)
\end{alltt}
\begin{verbatim}
## Importance of components:
##                          PC1   PC2    PC3
## Standard deviation     1.536 0.677 0.4282
## Proportion of Variance 0.786 0.153 0.0611
## Cumulative Proportion  0.786 0.939 1.0000
\end{verbatim}
\end{kframe}
\end{knitrout}

\end{frame}

\begin{frame}[fragile]
\frametitle{Algunos Ejemplos}
\framesubtitle{An\'alisis de Componentes Principales}
\begin{knitrout}\footnotesize
\definecolor{shadecolor}{rgb}{0.969, 0.969, 0.969}\color{fgcolor}\begin{kframe}
\begin{alltt}
\hlfunctioncall{biplot}(x)
\end{alltt}
\end{kframe}
\includegraphics[width=.5\linewidth]{figure/biplot} 

\end{knitrout}

\end{frame}

%% Analisis numerico (regress)
\begin{frame}[fragile]
\frametitle{Algunos Ejemplos}
\framesubtitle{Modelos lineales (MCO)}
\begin{knitrout}\scriptsize
\definecolor{shadecolor}{rgb}{0.969, 0.969, 0.969}\color{fgcolor}\begin{kframe}
\begin{alltt}
\hlfunctioncall{data}(USArrests)
x <- \hlfunctioncall{lm}(Murder ~ Assault + Rape, data = USArrests)
\hlfunctioncall{summary}(x)
\end{alltt}
\begin{verbatim}
## 
## Call:
## lm(formula = Murder ~ Assault + Rape, data = USArrests)
## 
## Residuals:
##    Min     1Q Median     3Q    Max 
## -4.867 -1.765 -0.375  1.303  7.886 
## 
## Coefficients:
##             Estimate Std. Error t value Pr(>|t|)    
## (Intercept)  0.41886    0.97618    0.43     0.67    
## Assault      0.04003    0.00609    6.58  3.6e-08 ***
## Rape         0.02514    0.05416    0.46     0.64    
## ---
## Signif. codes:  0 '***' 0.001 '**' 0.01 '*' 0.05 '.' 0.1 ' ' 1 
## 
## Residual standard error: 2.65 on 47 degrees of freedom
## Multiple R-squared: 0.645,	Adjusted R-squared: 0.63 
## F-statistic: 42.6 on 2 and 47 DF,  p-value: 2.76e-11
\end{verbatim}
\end{kframe}
\end{knitrout}

\end{frame}

%% Graficos con ggplot2
\begin{frame}[fragile]
\frametitle{Algunos Ejemplos}
\framesubtitle{ggplot2}
\begin{knitrout}\scriptsize
\definecolor{shadecolor}{rgb}{0.969, 0.969, 0.969}\color{fgcolor}\begin{kframe}
\begin{alltt}
\hlfunctioncall{library}(ggplot2)
m <- \hlfunctioncall{ggplot}(USArrests, \hlfunctioncall{aes}(x=Murder,y=Rape)) \hlcomment{# Def del objeto}
m + \hlfunctioncall{geom_line}() + \hlcomment{# Grafico de linea}
  \hlfunctioncall{geom_smooth}(method=\hlstring{"loess"}) + # IC
  \hlfunctioncall{labs}(title=\hlstring{"Violaciones vs \hlfunctioncall{Asesinatos} (IC)"}) # Etiqueta
\end{alltt}
\end{kframe}
\includegraphics[width=.5\linewidth]{figure/ggplot-demo} 

\end{knitrout}

\end{frame}

\frame{\maketitle
\begin{centering}
\scriptsize \tt
(presentaci\'on creada en R + knitr + \LaTeX)
\end{centering}
}

\end{document}
