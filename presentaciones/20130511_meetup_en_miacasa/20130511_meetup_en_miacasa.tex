


\documentclass{beamer}\usepackage{graphicx, color}
%% maxwidth is the original width if it is less than linewidth
%% otherwise use linewidth (to make sure the graphics do not exceed the margin)
\makeatletter
\def\maxwidth{ %
  \ifdim\Gin@nat@width>\linewidth
    \linewidth
  \else
    \Gin@nat@width
  \fi
}
\makeatother

\definecolor{fgcolor}{rgb}{0.2, 0.2, 0.2}
\newcommand{\hlnumber}[1]{\textcolor[rgb]{0,0,0}{#1}}%
\newcommand{\hlfunctioncall}[1]{\textcolor[rgb]{0.501960784313725,0,0.329411764705882}{\textbf{#1}}}%
\newcommand{\hlstring}[1]{\textcolor[rgb]{0.6,0.6,1}{#1}}%
\newcommand{\hlkeyword}[1]{\textcolor[rgb]{0,0,0}{\textbf{#1}}}%
\newcommand{\hlargument}[1]{\textcolor[rgb]{0.690196078431373,0.250980392156863,0.0196078431372549}{#1}}%
\newcommand{\hlcomment}[1]{\textcolor[rgb]{0.180392156862745,0.6,0.341176470588235}{#1}}%
\newcommand{\hlroxygencomment}[1]{\textcolor[rgb]{0.43921568627451,0.47843137254902,0.701960784313725}{#1}}%
\newcommand{\hlformalargs}[1]{\textcolor[rgb]{0.690196078431373,0.250980392156863,0.0196078431372549}{#1}}%
\newcommand{\hleqformalargs}[1]{\textcolor[rgb]{0.690196078431373,0.250980392156863,0.0196078431372549}{#1}}%
\newcommand{\hlassignement}[1]{\textcolor[rgb]{0,0,0}{\textbf{#1}}}%
\newcommand{\hlpackage}[1]{\textcolor[rgb]{0.588235294117647,0.709803921568627,0.145098039215686}{#1}}%
\newcommand{\hlslot}[1]{\textit{#1}}%
\newcommand{\hlsymbol}[1]{\textcolor[rgb]{0,0,0}{#1}}%
\newcommand{\hlprompt}[1]{\textcolor[rgb]{0.2,0.2,0.2}{#1}}%

\usepackage{framed}
\makeatletter
\newenvironment{kframe}{%
 \def\at@end@of@kframe{}%
 \ifinner\ifhmode%
  \def\at@end@of@kframe{\end{minipage}}%
  \begin{minipage}{\columnwidth}%
 \fi\fi%
 \def\FrameCommand##1{\hskip\@totalleftmargin \hskip-\fboxsep
 \colorbox{shadecolor}{##1}\hskip-\fboxsep
     % There is no \\@totalrightmargin, so:
     \hskip-\linewidth \hskip-\@totalleftmargin \hskip\columnwidth}%
 \MakeFramed {\advance\hsize-\width
   \@totalleftmargin\z@ \linewidth\hsize
   \@setminipage}}%
 {\par\unskip\endMakeFramed%
 \at@end@of@kframe}
\makeatother

\definecolor{shadecolor}{rgb}{.97, .97, .97}
\definecolor{messagecolor}{rgb}{0, 0, 0}
\definecolor{warningcolor}{rgb}{1, 0, 1}
\definecolor{errorcolor}{rgb}{1, 0, 0}
\newenvironment{knitrout}{}{} % an empty environment to be redefined in TeX

\usepackage{alltt}

\usetheme{Warsaw}
\usepackage[buttonsize=1em]{animate}
%\hypersetup{urlcolor=blue}

\title[useRChile]{Bienvenidos!}
\author[GGV]{George G. Vega}
\institute[useRchile]{Grupo de Usuarios de R en Chile}
\date{11 de Mayo, 2013}

\titlegraphic{\includegraphics[width=2cm]{usaR}\hspace*{4.75cm}~%
   \includegraphics[width=2cm]{RevolutionAnalyics}
}
\IfFileExists{upquote.sty}{\usepackage{upquote}}{}

\begin{document}
\begin{frame}
\maketitle
\end{frame}

\begin{frame}

\frametitle{Agenda}
\tableofcontents
\end{frame}

%%%%%%%%%%%%%%%%%%%%%%%%%%%%%%%%%%%%%%%%%%%%%%%%%%%%%%%%%%%%%%%%%%%%%%%%%%%%%%%%
\section{Un poco de historia}


\begin{frame}
\frametitle{Un poco de historia}
\framesubtitle{R, R Foundation y CRAN}
\begin{itemize}[<+->]
\item R es un lenguaje y ambiente para computaci\'on estad\'istica y gr\'aficos.
\item Es un proyecto GNU similar al lenguaje y ambiente de S el cual fu\'e implementado en los laboratorios Bell (AT\&T) por John Chambers y Cia.
\item R fue desarrollado por Robert Gentleman \& Ross Ihaka (R: A Language for Data Analysis and Graphics (1996)
Journal of Computational and Graphical Statistics)
\item CRAN (The Comprehensive R Network Archive) es una red de servidores Web y FTP alrededor del mundo que almacenan versiones de c\'odigo y documentaci\'on para R id\'entica.
\end{itemize}
\end{frame}

\begin{frame}[fragile]
\frametitle{The R Fundation for Statistical Computing}
The R fundation for Statistical Computing es una organizaci\'on sin fines de lucro trabajando
para el inter\'es com\'un. A sido construida por los miembros de R Development Core Team de
forma tal de:
\begin{itemize}[<+->]
\item Proveer soporte para R project y otras innovaciones en estad\'istica computacional. Creemos
que seR ha alcanzado una madurez y se ha convertido en una herramienta valiosa y nos gustar\'ia
asegurar su continuo desarrollo y el desarrollo de futuras innovaciones para la estad\'istica
computacional.
\item Proveer un punto de referencia para individuos, instituciones o empresas que quieran
aportar o interactuar  con la comunidad de desarrolladores de R.
\item Poseer y administrar el copyright de R y su documentaci\'on
\end{itemize}

R es una parte oficial del proyecto GNU de Free Software Foundation
\end{frame}

\begin{frame}
\frametitle{Desarrollo de R}
\framesubtitle{Task View de CRAN}

\begin{itemize}[<+->]
\item CRAN mantiene un listado extenso de \'areas de desarrollo de R, el Task View.
\item En ellas de detalla de manera exaustiva los paquetes tanto de R Dev Core
como de aquellos provistos por usuarios que apuntan en aquella direcci\'on
\item Algunas task views:
\begin{itemize}
\item Ecuaciones diferenciales
\item Econometr\'ia Computacional
\item Visualizaci\'on Gr\'afica
\item Computaci\'on de Alto rendimiento y en paralelo con R
\item Aprendizaje de m\'aquina y Estad\'istico
\item Estad\'isticas para las ciencias sociales
\end{itemize}
\end{itemize}
\end{frame}

%%%%%%%%%%%%%%%%%%%%%%%%%%%%%%%%%%%%%%%%%%%%%%%%%%%%%%%%%%%%%%%%%%%%%%%%%%%%%%%%



%%%%%%%%%% DONDE BUSCAR INFORMACION %%%%%%%%%%
\section{D\'onde buscar informaci\'on}
\frame{\frametitle{Agenda}\tableofcontents[currentsection]}

\begin{frame}
\frametitle{D\'onde buscar informaci\'on}
\framesubtitle{Tutoriales Online}
\begin{itemize}[<+->]
\item R wiki (oficial) \url{http://rwiki.sciviews.org/doku.php}
\item R-programming (wiki-book) \url{http://en.wikibooks.org/wiki/R_Programming}
\item Quick-R \url{http://www.statmethods.net/}
\item Flowing Data \url{http://flowingdata.com/category/tutorials/}
\end{itemize}
\end{frame}

\begin{frame}[fragile]
\frametitle{D\'onde buscar informaci\'on}
\framesubtitle{Libros}

\begin{figure}
\begin{minipage}[b]{0.65\textwidth}
\begin{itemize}%[<+->]
\item Norman Matloff, The Art of R Programming (2009) \url{http://heather.cs.ucdavis.edu/~matloff/132/NSPpart.pdf}
\item Introducci\'on a R \url{http://cran.r-project.org/doc/contrib/R-intro-1.1.0-espanol.1.pdf}
\item R para principiantes \url{http://cran.r-project.org/doc/contrib/rdebuts_es.pdf}
\end{itemize}
\end{minipage}
%\hfill
\begin{minipage}[b]{0.25\textwidth}
\centering
\includegraphics[width=1\textwidth]{theartofr}
\end{minipage}
\end{figure}
\end{frame}

\begin{frame}
\frametitle{D\'onde buscar informaci\'on}
\framesubtitle{Otros}
\begin{itemize}[<+->]
\item R-bloggers: Agregador de blogs sobre R \url{http://www.r-bloggers.com/}
\item Bioconductor: Herramientas para el an\'alisis gen\'etico (biol\'ogico) \url{http://www.bioconductor.org/}
\item CRAN: The Comprehensive R Archive \url{http://cran.r-project.org}
\item R Graph Gallery \url{http://gallery.r-enthusiasts.com/}
\item R Seek: Google de R \url{http://rseek.org/}
\item The R Journal \url{http://journal.r-project.org}
\end{itemize}
\end{frame}

\section{Algunos Ejemplos}
\frame{\frametitle{Agenda}\tableofcontents[currentsection]}

\begin{frame}[fragile]
\frametitle{Algunos Ejemplos}
\framesubtitle{animaciones}
\begin{knitrout}
\definecolor{shadecolor}{rgb}{0.969, 0.969, 0.969}\color{fgcolor}\begin{kframe}
\begin{alltt}
\hlfunctioncall{library}(animation)
\hlfunctioncall{demo}(\hlstring{"Mandelbrot"}, echo = FALSE, package = \hlstring{"animation"})
\end{alltt}
\end{kframe}



















\animategraphics[width=.45\linewidth,controls,loop]{2}{figure/animate-demo}{1}{20}

\end{knitrout}

\end{frame}


% Graficos 3D
\begin{frame}[fragile]
\frametitle{Algunos Ejemplos}
\framesubtitle{gr\'aficos 3D}
\begin{knitrout}\footnotesize
\definecolor{shadecolor}{rgb}{0.969, 0.969, 0.969}\color{fgcolor}\begin{kframe}
\begin{alltt}
\hlcomment{# Funcion}
fun <- \hlfunctioncall{function}(x, y) \{
    \hlfunctioncall{return}(\hlfunctioncall{sin}(x) - \hlfunctioncall{cos}(x))
\}

\hlcomment{# Graficando}
x <- y <- \hlfunctioncall{seq}(0, 2 * pi, pi/18)
\hlfunctioncall{persp}(x, y, z = \hlfunctioncall{outer}(x, y, fun), col = \hlstring{"blue"})
\end{alltt}
\end{kframe}
\includegraphics[width=.45\linewidth]{figure/persp-demo} 

\end{knitrout}

\end{frame}

\def\title{Gracias!}
\frame{
\maketitle
\begin{centering}
\scriptsize \tt
(presentaci\'on creada en R + knitr + \LaTeX)
\end{centering}
}

\end{document}
